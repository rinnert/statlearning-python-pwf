\documentclass[mathserif, aspectratio=169]{beamer}
%
%%%%%%%%%%%%%%%%%%%%%%%%%%%%%%%%%%%%%%%%%%%%%%%%%%%%%%%%%%%%%%%%%%%%%%%%
% need to split the includes to make spell checking work.
\usepackage{arev, arevmath}
\usepackage[scaled]{cabin}
\usepackage[T1]{fontenc}
\usepackage[super]{nth}
\usepackage{pifont}
\usepackage{wasysym}
\usepackage{tabularx}
\usepackage{array}
\usepackage{booktabs}
\usepackage{boldline}
\usepackage{colortbl}
%\usepackage{amsmath}
\usepackage{bm}
\usepackage{tcolorbox}
\usepackage{adjustbox}
\usepackage{minibox}
\usepackage{makecell}
\usepackage{adjustbox}
\usepackage{textcomp}
\usepackage[absolute,overlay]{textpos}
\setlength{\TPHorizModule}{1mm}%
\setlength{\TPVertModule}{1mm}%
\tcbuselibrary{skins}

\makeatletter
\newcommand{\antsize}{\@setfontsize{\antsize}{4pt}{4pt}}
\makeatother
\newcommand{\at}{\makeatletter @\makeatother}

\newcommand{\cmark}{\ding{51}}%
\newcommand{\bottomline}[1]{\vskip0pt plus 1fill{\alert{#1}\phantom{g}\vskip 1.0mm}}

\newcommand{\Quote}[2]{%
	\begin{center} 
		\begin{minipage}{0.7\textwidth} 
			\hrule
			\vskip 3mm
			\emph{{\color{ICTPblue} ``#1''}
			
			~~~~ {\color{ICTPorange} --- #2}}
			\vskip 3mm
			\hrule
			\vskip 2mm
		\end{minipage}
	\end{center}}


\mode<presentation>%
{
	\usetheme{default}
	%\usetheme[width=2.5cm]{PaloAlto}
	\usecolortheme{dove}
	\useoutertheme{infolines}
	% oder auch nicht

	% ICTP Colors
	\definecolor{ICTPblue}{RGB}{37,86,162} % 0x255682
	\definecolor{ICTPorange}{RGB}{255,130,0} % 0xff8200
	\definecolor{ICTPgreen}{RGB}{0,100,0}
	\definecolor{ICTPdark}{RGB}{80,80,80} % 0x505050
	\definecolor{ICTPlight}{RGB}{120,120,120}
	\definecolor{ICTPbrown}{RGB}{178,91,0}

	\definecolor{codebg}{rgb}{0.95,0.95,0.95}

	% Color theme
	\setbeamercolor{alerted text}{fg=ICTPorange}
	\setbeamercolor{frametitle}{fg=ICTPblue}
	\setbeamercolor{title}{fg=ICTPblue}
	\setbeamercolor{subtitle}{fg=ICTPorange}
	\setbeamercolor{normal text}{fg=ICTPdark}
	\setbeamercolor{author in foot}{fg=ICTPblue}
	\setbeamercolor{item}{fg=ICTPblue}
	\setbeamercolor{footline}{fg=ICTPblue}
	%\setbeamercolor{item projected}{bg=ICTPorange}
	%\setbeamercolor{item projected}{fg=white}

	\setbeamertemplate{headline}
	{}
	\setbeamertemplate{frametitle}
	{
		%\textbf{{\insertframetitle\phantom{g}}}\\
		%\textbf{\insertframetitle\phantom{g}}\\
		\textbf{\underline{\insertframetitle\phantom{g}}}\\
		%\textbf{\underline{\insertframetitle}}\\
		\vskip 1.0mm
		%{\color{UOLgold}\hrule height 2pt}
		%\par
	}
	\addtobeamertemplate{frametitle}{}{\vspace{-1em}}
	\setbeamertemplate{footline}{
		{%
			\textbf{ \hskip 3.0mm\insertshorttitle\phantom{.}---\phantom{.}\insertshortinstitute\hfill\insertframenumber\,/\,\inserttotalframenumber\hskip 3.0mm} 
		}
	}

	\setbeamertemplate{navigation symbols}{}%remove navigation symbols
	\setbeamertemplate{itemize items}[circle]
	\setbeamertemplate{enumerate items}[fg=ICTPblue]
	\setbeamercolor{itemize items}{fg=ICTPblue}
	\setbeamercolor{sidebar}{bg=ICTPblue}
	\setbeamercolor{title in sidebar}{fg=ICTPorange}
	\setbeamercolor{author in sidebar}{fg=ICTPorange}
	\setbeamercolor{section in sidebar}{fg=ICTPorange}
}

%\input{tikz/common-styles}

\usepackage{graphicx}
\usepackage[latin1]{inputenc}

\graphicspath{{../figs/}{../figs/common/}{../figs/islr/}}

\title[Statistical Learning] % (optional, nur bei langen Titeln n�tig)
{\textbf{Introduction to Statistical Learning\\ {\it with applications in Python}}\\%
		\href{www.statlearning.com}%
		{\tiny\it Based on ``Introduction to Statistical Learning, with applications in R'' by Gareth James, Daniela Witten, Trevor Hastie, Robert Tibishirani}\vspace{2em}}
		\vspace{-2.5cm}{}


		\author{\href{mailto:?to=Kurt Rinnert <kurt.rinnert@cern.ch>&subject=PWF Statistical Learning}{Kurt Rinnert}}

\institute[{\href{https://www.ictp.it/physics-without-frontiers.aspx}{Physics Without Frontiers} --- \href{https://www.ictp.it/}{ICTP}}] % (optional)
{\color{ICTPblue}\bfseries \href{https://www.ictp.it/physics-without-frontiers.aspx}{Physics Without Frontiers}\\\vspace{1mm}%
\href{https://www.ictp.it/}{\includegraphics[width=0.20\textwidth]{common/ICTP-logo-full-trans.png}}\\%
\href{https://www.liverpool.ac.uk/physics/}{\includegraphics[width=0.2\textwidth]{common/uol_logo.png}}}

\date{}

\titlegraphic{
	\texorpdfstring{\vspace{-2.8cm}}{}
	 \begin{minipage}[b][1.3cm][b]{0.26\textwidth}\color{ICTPlight}\antsize
		Copyright \textcopyright~2019\\
		\href{mailto:?to=Kurt Rinnert <kurt.rinnert@cern.ch>&subject=PWF Statistical Learning}{Kurt Rinnert <kurt.rinnert{\tt @}cern.ch>},
		\href{mailto:?to=Kate Shaw <kshaw@ictp.it>&subject=PWF Statistical Learning}{Kate Shaw <kshaw{\tt @}ictp.it>}\\
		Copying and distribution of this file, with or without modification,
		are permitted in any medium without royalty provided the copyright
		notice and this notice are preserved.  This file is offered as-is,
		without any warranty.


		Some of the figures in this presentation are taken from ``An Introduction to
		Statistical Learning, with applications in R''  (Springer, 2013) with
		permission from the authors: G. James, D. Witten,  T. Hastie and R. Tibshirani 
	 \end{minipage}\hspace{10cm}
}


\addtocounter{framenumber}{-1}

% nicer table row separation
\renewcommand{\arraystretch}{1.2}

% color boxes
\newcommand{\tabboxset}{\tcbset{enhanced, nobeforeafter, boxrule=0pt, boxsep=0pt, colback=codebg, colframe=codebg, coltext=ICTPdark, rounded corners, arc=4pt, fonttitle={\bfseries\tiny}}}
\newcommand{\codeboxset}{\tcbset{enhanced, nobeforeafter, boxrule=0pt, boxsep=0pt, colback=codebg, colframe=codebg, coltext=ICTPdark, rounded corners, arc=4pt, fonttitle={\bfseries\tiny}}}

\newcommand{\orange}{\color{ICTPorange}}
\newcommand{\blue}{\color{ICTPblue}}
\newcommand{\dark}{\color{ICTPdark}}
\newcommand{\R}{\mathbb{R}}
\newcommand{\dat}[1]{{\footnotesize\tt\orange #1}}
\newcommand{\e}[1]{\emph{#1}}

\makeatletter
\newcommand{\includegraphicsdpi}[3]{%
	\pdfimageresolution=#1%
	\includegraphics[#2]{#3}%
	\pdfimageresolution=72%
}

\newenvironment{blurb}%
	{\begin{center}\begin{minipage}{0.6\textwidth}\footnotesize}
	{\end{minipage}\end{center}}

\newenvironment{popblock}[2]%
	{\begin{center}\begin{minipage}{#1}\footnotesize
		\begin{tcolorbox}[colframe=codebg, colback=white, colupper=ICTPdark, title={\normalsize\bfseries\blue #2}]}
	{\end{tcolorbox}\end{minipage}\end{center}}
\makeatother

\subtitle{\bfseries%
  {Linear Regression, Part 3}\\%
  {\tiny\it qualitative predictors, interaction \& non-linear extensions, outliers, collinearity}\\%
}
\begin{document}
\frame[plain]{
	\vskip 1.0mm
	\titlepage
	\vskip 1.0mm
}


\begin{frame}{Abstract}

	\begin{blurb}
		Linear models are an important topic in statistical learning.  

		The true relationships between predictors and responses are rarely linear.
		But linear models often provide reasonable approximation. They provide
		high interpretability and have low variance, mitigating the risk of over-fitting.
		Linear models can be extended to include (some) non-linear relationships. 

		Linear models also provide an excellent baseline to compare other models against: if 
		our sophisticated model does not do much better than a linear model we might consider
		trading some bias for lower variance.
	\end{blurb}
\end{frame}

\begin{frame}{Overview}
	\begin{itemize}
		\item Qualitative versus quantitative predictors.
		\item Interactions among predictors.
		\item Non-linear extensions to the linear model.
		\item Outliers \& high leverage points.
		\item Collinearity.
	\end{itemize}
	\bottomline{This will conclude our long journey through linear regression.}
\end{frame}

\begin{frame}{Example: Credit Data Set}
	\begin{columns}
		\begin{column}{0.5\textwidth}
			\begin{itemize}
				\item The plot on the right shows the \e{quantitative} variables
					in the data set.
				\item The dataset also contains \e{qualitative} predictors:
				\item[] \dat{gender}: \val{Male}, \val{Female}
				\item[] \dat{married}: \val{Yes}, \val{No}
				\item[] \dat{student}: \val{Yes}, \val{No}
				\item[] \dat{ethnicity}: \val{Asian}, \val{African American}, \val{Caucasian}
			\end{itemize}
		\end{column}
		\begin{column}{0.5\textwidth}
			\vspace{-10mm}
			\begin{center}
				\includegraphics[width=0.95\textwidth]{3_6s}
			\end{center}
		\end{column}
	\end{columns}
	\bl{We need to somehow \e{encode} the qualitative predictors.}
\end{frame}

\begin{frame}{Qualitative Predictors with Two Levels}
	\begin{itemize}
		\item The variables \dat{gender}, \dat{student} and \dat{married} are qualitative
			predictors with two levels.
		\item Suppose we are interested in whether \dat{gender} influences \dat{balance}.
		\item We can define a \e{dummy variable} to encode the gender:
			\[
				x_i =
				\begin{cases}
					1 & \text{if $i$th person is female} \\
					0 & \text{if $i$th person is male} \\
				\end{cases}
			\]
		\item This results in the model
			\[
				y_i = \beta_0 + \beta_1 x_i + \epsilon_i =
				\begin{cases}
					\beta_0 + \beta_1 + \epsilon_i & \text{if $i$th person is female} \\
					\beta_0 + \epsilon_i & \text{if $i$th person is male} \\
				\end{cases}
			\]
	\end{itemize}
	\bl{The distinction is important for the interpretation of the model.}
\end{frame}

\begin{frame}{Qualitative Predictors with Two Levels}
	\begin{popblock}{0.8\textwidth}{Model Interpretation}
			\begin{align*}
				\beta_0 &: \text{average \dat{{\tiny balance}} among males}\\
				\beta_0 + \beta_1 &: \text{average \dat{{\tiny balance}} among females}\\
				\beta_1 &: \text{average \dat{{\tiny balance}} difference between females and males}\\
			\end{align*}
	\end{popblock}
	\begin{popblock}{0.8\textwidth}{Fit Result}
		\begin{tabular}[h]{lrrrr}
			{} & {\blue Coefficient} & {\blue Std. Error} & {\blue $t$-statistic} & {\blue $p$-value} \\
			\dat{Intercept} & $509.80$ & $33.13$ & $15.389$ & $< 0.0001$ \\
			\dat{gender[Female]} & $19.73$ & $46.05$ & $0.429$ & $0.6690$ \\
		\end{tabular}
	\end{popblock}
	\bl{The observed average difference of \$19.73 is \e{not} significant.}
\end{frame}

\begin{frame}{Qualitative Predictors with Two Levels}
	\begin{itemize}
		\item We can also encode the gender differently:
			\[
				x_i =
				\begin{cases}
					1 & \text{if $i$th person is female} \\
					-1 & \text{if $i$th person is male} \\
				\end{cases}
			\]
		\item This results in the model
			\[
				y_i = \beta_0 + \beta_1 x_i + \epsilon_i =
				\begin{cases}
					\beta_0 + \beta_1 + \epsilon_i & \text{if $i$th person is female} \\
					\beta_0 - \beta_1 + \epsilon_i & \text{if $i$th person is male} \\
				\end{cases}
			\]
	\end{itemize}
	\bl{This leads to a different interpretation of the coefficients.}
\end{frame}

\begin{frame}{Qualitative Predictors with Two Levels}
	\begin{popblock}{0.8\textwidth}{Model Interpretation}
			\begin{align*}
				\beta_0 &: \text{overall average \dat{{\tiny balance}}, disregarding gender}\\
				\beta_1 &: \text{amount  above (below) average for females (males)}\\
			\end{align*}
	\end{popblock}
	\begin{popblock}{0.8\textwidth}{Fit Result}
		\begin{tabular}[h]{lrrrr}
			{} & {\blue Coefficient} & {\blue Std. Error} & {\blue $t$-statistic} & {\blue $p$-value} \\
			\dat{Intercept} & $519.67$ & $23.03$ & $22.569$ & $< 0.0001$ \\
			\dat{gender} & $9.87$ & $23.03$ & $0.429$ & $0.6690$ \\
		\end{tabular}
	\end{popblock}
	\bl{Note that the fit is essentially the same as before.}
\end{frame}
\end{document}
